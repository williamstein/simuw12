\documentclass[11pt]{article}
\pagestyle{empty}
\title{Exercise: 1 is not a Congruent Number}
\author{William stein}
\newcommand{\Z}{\mathbf{Z}}
\newcommand{\ord}{{\rm ord}}
%\textheight=1.1\textheight
%\voffset=-1in
\begin{document}
\vspace{-5em}

\maketitle

If you can do this exercise, you will prove that $1$ is not a
congruent number, hence that the area of a rational right triangle can
not be a perfect square.  Along the way, you will also prove Fermat's
Last Theorem for exponent $4$ (and something even stronger).


\begin{enumerate}
\item\label{ex:pythag} Suppose $a$, $b$, $c$ are relatively prime
  integers with $a^2+b^2=c^2$ (relatively prime means that no prime
  number simultaneously divides all three of $a,b,c$).  Then there
  exist integers $x$ and $y$ with $x>y$ relatively prime such that
  $c=x^2+y^2$ and either $a=x^2-y^2$, $b=2xy$ or $a=2xy$, $b=x^2-y^2$.
  [Hint: use that the unit circle $X^2+Y^2=1$ is parametrized by
  $X=\frac{1-t^2}{1+t^2}$, $Y=\frac{2t}{1+t^2}$.]

\item \label{ex:flt4} Fermat's Last Theorem for exponent $4$ asserts
  that any solution to the equation $x^4+y^4=z^4$ with $x,y,z\in\Z$
  satisfies $xyz=0$.  Prove Fermat's Last
  Theorem for exponent $4$, as follows:
\begin{enumerate}
\item Show that if the equation $x^2+y^4=z^4$ has no integer solutions
  with $xyz\neq 0$, then Fermat's Last Theorem for exponent~$4$ is
  true.
\item Show that if $n,k,m$ are integers with $n^2+k^4=m^4$ and $p$ is
  a prime that divides both $k$ and $m$, then $p^2$ divides $n$.  Thus
  by dividing both sides by $p^4$, we see that there exists an integer
  solution with $n,k,m$ not all divisible by $p$.
\item(*) Prove that $x^2+y^4=z^4$ has no integer solutions with $xyz\neq
  0$ as follows.  Suppose $n^2+k^4=m^4$ is a solution with $m>0$
  minimal among all solutions.  Show that there exists a solution with
  $m$ smaller using Exercise~\ref{ex:pythag} (consider two cases as below).
This is called Fermat's ``method of infinite descent.''
\begin{enumerate}
\item {\bf Case one:} $m^2=x^2+y^2, n=2xy, k^2=x^2-y^2$.  [Hint:
  Consider $m^2k^2$, which should make the solution clear in this
  case.]
\item {\bf Case two:} $m^2=x^2+y^2, n=x^2-y^2, k^2=2xy$.  
[Hint: Since $2xy$ is a perfect square, we have $x=2u^2$ and $y=v^2$ (the
other way around is similar).
From the Pythagorean triple $m^2=x^2+y^2$, we have
$x=2rs$ and $y=r^2-s^2$, and $m=r^2+s^2$.
Since $2u^2=2rs$, we have $r=g^2$ and $s=h^2$.
Substituting these, along with $y=v^2$, into
$y=r^2-s^2$ gives $v^2=g^4-h^4$, hence $v^2+h^4=g^4$, with $g<m$.]
\end{enumerate}
\end{enumerate}

\item\label{ex:cong1}(*) Prove that $1$ is not a congruent number by
  showing that the elliptic curve $y^2=x^3-x$ has no rational
  solutions except $(0,\pm 1)$ and $(0,0)$, as follows:
\begin{enumerate}
\item Write $y=\frac{p}{q}$ and $x=\frac{r}{s}$, where $p,q,r,s$ are
all positive integers and $\gcd(p,q)=\gcd(r,s)=1$.  Prove that $s\mid
q$, so $q=sk$ for some $k\in\Z$.
\item Prove that $s=k^2$ by substituting $y=p/(sk), x=r/s$ into $y^2=x^3-x$
and putting both sides of the equation in lowest terms. Substitute $s-k^2$
to see that $p^2=r^3-rk^4$.
\item Prove that $r$ is a perfect square by supposing that there is a
  prime~$\ell$ such that $\ord_{\ell}(r)$ is odd (i.e., the power of $\ell$
in the factorization of $r$ is odd), and analyzing
  $\ord_{\ell}$ of both sides of $p^2=r^3-rk^4$.
\item Write $r=m^2$, and substitute to
see that $p^2=m^6-m^2k^4$. Prove that $m\mid p$.  
\item Divide through by $m^2$ and deduce a contradiction 
to Exercise~\ref{ex:flt4}.
\end{enumerate}

\end{enumerate}

\end{document}

%%% Local Variables: 
%%% mode: latex
%%% TeX-master: t
%%% End: 
